\cventry{2021--}{Independent Researcher}{Rakta Network Oy, Finland}
{}{}{\textbf{Independent Researcher, consultant }
    Offering consultant services for mathematical and scientific software and tooling related to imaging, image processing, and analysis, particularly large-scale volumetric data.}
\cventry{2018--2021}{Research Scientist}{University of California, San Francisco}
{}{}{\textbf{Research Scientist at the National Center for X-ray Tomography, provide computational support for internal and collaborative research projects.}
    Primary responsibility for maintaining and developing the automatic image processing pipeline at NCXT and providing computational tools for image analysis. Exploring the prospects of modern machine learning in soft x-ray tomography.}
\cventry{2016--2018}{Postdoctoral Scholar}{University of California, San Francisco}
{}{}{\textbf{Developing methods and software for improved data reconstruction and processing}
    Main responsibilities include the maintaining and developing the image processing pipeline at National Center for X-ray Tomography. This includes the reconstruction process from the raw images to the final volumetric representations for both the fluorescence and x-ray microscope as well as developing various methods for analyzing the structure of biological samples.}
%
\cventry{2011--2016}{Doctoral Student}{University of Jyväskylä, Finland}
{}{}{\textbf{Theoretical and numerical work on random deposition networks. }
    During this period we extended on the work regarding the structure of such random networks by considering, in more detail, the effect of the steric hindrance between the constituents. The theoretical framework was backed up by both numerical simulations as well as experimental results.}
%
\cventry{2010--2011}{Research associate}{University of Jyväskylä, Finland}
{}{}{\textbf{Analyzing the structure of paper and cardboard using micro and nano scale X-ray Computed Tomography. }
    The main focus was on the development of quantitative tools for analysis properties of cardboard from their 3d tomographic images.}